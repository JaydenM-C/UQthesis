% *************** Document style definitions ***************

% ******************************************************************
% This file defines the document design.
% Usually it is not necessary to edit this file, but you can use it to change aspects of the design if you want.
% ******************************************************************

%------------------------------------------------------------------------------%
%----------------------------LOAD PACKAGES-------------------------------------%
%------------------------------------------------------------------------------%

% Feel free to alter/add to these packages as you need.
% ******************* Load packages *******************
%Miscellaneous.
\usepackage{cite}									%Allows abbreviated numerical citations.
\usepackage[figuresright]{rotating}	%Allows large tables to be rotated to landscape.
\usepackage{pdfpages}							%Allows you to include full-page pdfs.
\usepackage{wrapfig}							%Lets you wrap text around figures.
%Maths stuff.
\usepackage{bm} 									%Bolded maths characters.
\usepackage{upgreek}							%Upright Greek characters.
\usepackage{dsfont}								%Double-struck fonts.
%\usepackage{simplewick}						%For typesetting Wick contractions.
\usepackage{mathtools}						%Can be used to fine-tune the maths presentation.	
%Text packages.
\usepackage{framed}								%For boxed text.
\usepackage{microtype}						%pdfLaTeX will fix your kerning.
\usepackage{marvosym}							%Include symbols (like the Euro symbol, etc.).
%Figures.
	\usepackage{color}							%Nice for scalable pdf graphics using InkScape.
	\usepackage{transparent}				%Nice for scalable pdf graphics using InkScape.
\usepackage{placeins}							%Lets you put in a \FloatBarrier to stop figures floating past this command.
%Lists.
\usepackage{mdframed,mdwlist} 		%Use these for nice lists (less white space).

%------------------------------------------------------------------------------%
%---------------------MACROS-----THE-BLACK-------------------------------------%
%------------------------------------------------------------------------------%

%Define a bunch of macros that implement Latin abbreviations.
%COMMENT OUT OR DELETE IF UNDESIRED.
\newcommand{\via}{\textit{via}} %Italicised via.
\newcommand{\ie}{\textit{i.e.}} %Literally.
\newcommand{\eg}{\textit{e.g.}} %For example.
\newcommand{\etc}{\textit{etc.}} %So on...
\newcommand{\vv}{\textit{vice versa}} %And the other way around.
\newcommand{\viz}{\textit{viz}.} %Resulting in.
\newcommand{\cf}{\textit{cf}.} %See, or 'consistent with'.
\newcommand{\apr}{\textit{a priori}} %Before the fact.
\newcommand{\apo}{\textit{a posteriori}} %After the fact.
\newcommand{\vivo}{\textit{in vivo}} %In the flesh.
\newcommand{\situ}{\textit{in situ}} %On location.
\newcommand{\silico}{\textit{in silico}} %Simulation.
\newcommand{\vitro}{\textit{in vitro}} %In glass.
\newcommand{\vs}{\textit{versus}} %James \vs{} Pete.
\newcommand{\ala}{\textit{\`{a} la}} %In the manner of...
\newcommand{\apriori}{\textit{a priori}} %Before hand.
\newcommand{\etal}{\textit{et al.}} %And others, with correct punctuation.
\newcommand{\naive}{na\"\i{}ve} %Queen Amidala is young and \naive{}.

% *************** End of document style definition ***************